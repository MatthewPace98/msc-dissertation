\chapter{Introduction}

\textbf{Note that you may have multiple \texttt{{\textbackslash}include} statements here, e.g.\ one for each subsection.}

What is AML?
Describe the biology
Define: Bioinformatics, RNAseq, Leukaemia, gene expression, RNA
The model cell line


\section{Acute Myeloid Leukaemia}
Leukaemia is a group of diseases of the hematopoietic system, characterised according to cell phenotype, disease progression and responsiveness to treatment. A multinational effort was made to create a standard classification system for leukaemias, which cumulated into the French-American-British \ac{FAB} classification system. Since its initial proposal \citep{bennett1976proposals} and after , it has become widely accepted .


\section{The Role of RNA in Living Organisms}
%The central dogma of molecular biology



\section{Next Generation Sequencing}
%Low complexity regions
%Sequencing software which cannot determine the nucleotide in a particular position, often represent the ambiguous base as 'N'. 
%Library preperation
%Adapter sequences
%Demultiplexing

\subsection{RNA sequencing}



\section{RNA sequencing for the detection of differentiation}
%What is RNA-seq
%Go into the molecular biology a bit  (Central dogma of molecular biology)
%Distiguish between earlier methods(microarrays)
%First generation (Sanger sequences)
%Second generation (NGS, massivly parralel format so millions of sequencing reactions can run simaltaneously)
%Third-generation sequencing(also massively parallel but have templates)

\section{Motivation} 
% why is this a non trivial problem
%Something about AML death statistics
%Something about current treatments
%To improve treatment
%Advancement of RNA-seq technology


\section{Aims and Objectives} 

The general aim of this dissertation is to observe the 
effects of this phenolic treatment on the HL-60’s
transcriptome through bulk RNA-Seq analysis. To achieve
this, six objectives must be attained:
(i.) Determine which bioinformatics tools are best 
suited to be used in this RNA-Seq pipeline given the 
current dataset.
(ii.) Perform quality control checks on the data files at 
various stages of analysis.
(iii.) Align the reads to a recent human reference genome 
release.
(iv.) Normalise the data to account for differences 
between samples and between genes.
(v.) Conduct differential expression analysis.
(vi.) Visualise and interpret the gene expression data to 
assess the level of differentiation caused by phenol 
treatment occurring across the three time points.



\begin{figure}[ht!] % supposedly places it here ...
  \centering
  \includegraphics[width=0.6\linewidth]{test_image_goku}
  \caption[This is the short caption for List of Figures]{A test figure.  This caption is huge, but in the list of figures only the smaller version in the square brackets will appear.\index{Goku il-king}}
  \label{fig:test1}
\end{figure}

A test figure is shown in Figure~\ref{fig:test1}.

\section{Proposed Solution} 
Another weapon in our arsenal to use in the fight against AML (abrreveation)
RNA-seq pipeline and analysis

\begin{figure}[!ht]
    \centering
    \subbottom[Goku]{\includegraphics[width=0.3\textwidth]{test_image_goku}}\qquad
    \subbottom[More Goku]{\includegraphics[width=0.3\textwidth]{test_image_goku}}%
    \caption[Short Caption]{The same super saiyan. Two times.} 
    \label{fig:test2}
\end{figure}

Two figures shown side by side are shown in Figure~\ref{fig:test2}.
\section{Results}
Summary of the results.


\section{Document Structure}
